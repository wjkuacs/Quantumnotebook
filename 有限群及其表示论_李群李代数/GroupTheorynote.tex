\documentclass{article}
\usepackage{ctex}
\usepackage{amsmath}
\usepackage{amssymb}
\usepackage{amsthm}
\usepackage{amsfonts}
\usepackage{physics}
\usepackage{geometry}
\usepackage{graphicx}
\usepackage{pgfplots}
\pgfplotsset{compat=1.18}
\usepackage{caption}
\usepackage{subcaption}
\usepackage{hyperref}
\usepackage{float}
\usepackage{fancyhdr}

\pagestyle{fancy}
\fancyhead[L]{
    \begin{minipage}[c]{0.06\textwidth}
        \centering
        \includegraphics[height=7.5mm]{院徽.jpg}
    \end{minipage}
    \begin{minipage}[c]{0.4\textwidth}
    Note of Group Theory
    \end{minipage}
}
\fancyfoot[C]{\thepage}

%\setlength{\textwidth}{1.2\textwidth}
%\setlength{\textheight}{0.8\textheight}

\newtheorem{theorem}{定理}
\newtheorem{definition}{定义}
\newtheorem{example}{例}
\newtheorem{solution}{解}
\newtheorem{question}{题目}



\geometry{
paper = a4paper,    %纸张类型
top = 3cm,          %上页边距
bottom = 3cm,       %下页边距
left = 3cm,         %左页边距
right = 3cm         %右页边距
}

\newcommand{\ds}{\displaystyle}
\newcommand{\bb}[1]{\mathbb{#1}}
\newcommand{\h}[1]{\hat{#1}}
\newcommand{\pmtwo}[4]{\begin{pmatrix}#1&#2\\#3&#4\end{pmatrix}}
\newcommand{\pmthree}[9]{
    \begin{pmatrix}
        #1&#2&#3\\
        #4&#5&#6\\
        #7&#8&#9
    \end{pmatrix}
}
\newcommand{\vmtwo}[4]{\begin{vmatrix}#1&#2\\#3&#4\end{vmatrix}}
\newcommand{\vmthree}[9]{
    \begin{vmatrix}
        #1&#2&#3\\
        #4&#5&#6\\
        #7&#8&#9
    \end{vmatrix}
}

\newcommand{\vmthreedot}[9]{
    \begin{vmatrix}
        #1&#2&\cdots&#3\\
        #4&#5&\cdots&#6\\
        \vdots&\vdots&\ddots&\vdots\\
        #7&#8&\cdots&#9
    \end{vmatrix}
}

\newcommand{\expectation}[1]{\langle #1 \rangle}

\newcommand{\Da}[2]{\frac{\partial}{\partial#2}#1}

\newcommand{\D}[2]{\frac{d}{d#2}#1}

\newcommand{\id}[1]{\ket{#1}\bra{#1}}





\title{Note of Group Theory}
\author{Kaiser\\University of South China}




\begin{document}

\maketitle




\begin{abstract}
    \normalsize
    这一篇笔记是本科阶段学习群论的笔记。
\end{abstract}

\begin{center}
    \large
    \begin{thebibliography}{99} 
        \bibitem{1} Joseph J. Rotman, An Introduction to the Theory of Groups
        \bibitem{2} 周彬, Lie 群与 Lie 代数
        \bibitem{3} Pierre Ramond, Group Theory : A Physicist's Survey
        \bibitem{4} 崔建伟, 群论讲义
        \bibitem{5} A. Das and S. Okubo, Lie Groups and Lie Algebras for Physicists
    \end{thebibliography}
\end{center}


\begin{center}
    \tableofcontents
\end{center}

\newpage



\section{有限群(Finite Group)}


\subsection{群与群乘法}

废话也就不多说,我们直接给出群的定义,并在之后讲解一下为什么要这样去定义

\begin{definition}[Definition of a group]
    \ \\ $\indent$Suppose we give a set $\mathbb{G}=\{g_0,g_1,g_2,\cdots\}$ and an operator $\cdot$, if they satisfy the following properties
    \begin{enumerate}
        \item \textbf{Closure} For every ordered pair of elements, $g_i$ and $g_ j$, there exists a unique element
        \begin{equation}
            a_i\cdot a_j = a_k
        \end{equation}
        for any three $i, j, k$.
        \item \textbf{Associativity} The $\cdot$ operation is associative
        \begin{equation}
            g_i\cdot\left(g_j\cdot g_k\right) = \left(g_i\cdot g_j\right)\cdot g_k
        \end{equation}
        \item \textbf{Unit element} The set G contains a unique element e(here we will use $g_0$ to symbol it) such that
        \begin{equation}
            g_0\cdot g_i = g_i \cdot g_0 =g_i
        \end{equation}
        for all i. In particular, this means that
        \begin{equation*}
            g_0 \cdot g_0 = g_0.
        \end{equation*}
        \item \textbf{Inverse element} Corresponding to every element $g_i$, there exists a unique element
        of G, the inverse $(g_i)^{−1}$ such that
        \begin{equation}
            g_i \cdot (g_i)^{-1}=(g_i)^{-1} \cdot g_i =g_0
        \end{equation}
        In particular,this means that
        \begin{equation*}
            g_0 = (g_0)^{-1}
        \end{equation*}
    \end{enumerate}

    then we called the set $\mathbb{G}$ a Group.
\end{definition}

同时,我们根据集合$\mathbb{G}$的元素个数



















\newpage
\subsection{群同态(Group Homomorphisms)与群的直乘}


\subsection{群的线性表示理论}









\section{表示论}



\section{李群(Lie Group)}




\section{李代数(Lie Algebra)}



\end{document}